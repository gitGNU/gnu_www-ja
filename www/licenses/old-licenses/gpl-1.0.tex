\documentclass[]{article}
\usepackage{lmodern}
\usepackage{amssymb,amsmath}
\usepackage{ifxetex,ifluatex}
\usepackage{fixltx2e} % provides \textsubscript
\ifnum 0\ifxetex 1\fi\ifluatex 1\fi=0 % if pdftex
  \usepackage[T1]{fontenc}
  \usepackage[utf8]{inputenc}
\else % if luatex or xelatex
  \ifxetex
    \usepackage{mathspec}
  \else
    \usepackage{fontspec}
  \fi
  \defaultfontfeatures{Ligatures=TeX,Scale=MatchLowercase}
\fi
% use upquote if available, for straight quotes in verbatim environments
\IfFileExists{upquote.sty}{\usepackage{upquote}}{}
% use microtype if available
\IfFileExists{microtype.sty}{%
\usepackage{microtype}
\UseMicrotypeSet[protrusion]{basicmath} % disable protrusion for tt fonts
}{}
\usepackage{hyperref}
\hypersetup{unicode=true,
            pdfborder={0 0 0},
            breaklinks=true}
\urlstyle{same}  % don't use monospace font for urls
\IfFileExists{parskip.sty}{%
\usepackage{parskip}
}{% else
\setlength{\parindent}{0pt}
\setlength{\parskip}{6pt plus 2pt minus 1pt}
}
\setlength{\emergencystretch}{3em}  % prevent overfull lines
\providecommand{\tightlist}{%
  \setlength{\itemsep}{0pt}\setlength{\parskip}{0pt}}
\setcounter{secnumdepth}{0}
% Redefines (sub)paragraphs to behave more like sections
\ifx\paragraph\undefined\else
\let\oldparagraph\paragraph
\renewcommand{\paragraph}[1]{\oldparagraph{#1}\mbox{}}
\fi
\ifx\subparagraph\undefined\else
\let\oldsubparagraph\subparagraph
\renewcommand{\subparagraph}[1]{\oldsubparagraph{#1}\mbox{}}
\fi

\date{}

\begin{document}

\subsection{GNU GENERAL PUBLIC LICENSE}

Version 1, February 1989

\begin{verbatim}
Copyright (C) 1989 Free Software Foundation, Inc. 51 Franklin St,
Fifth Floor, Boston, MA 02110-1301 USA Everyone is permitted to copy
and distribute verbatim copies of this license document, but changing
it is not allowed.
\end{verbatim}

\subsubsection{Preamble}

The license agreements of most software companies try to keep users at
the mercy of those companies. By contrast, our General Public License is
intended to guarantee your freedom to share and change free software--to
make sure the software is free for all its users. The General Public
License applies to the Free Software Foundation's software and to any
other program whose authors commit to using it. You can use it for your
programs, too.

When we speak of free software, we are referring to freedom, not price.
Specifically, the General Public License is designed to make sure that
you have the freedom to give away or sell copies of free software, that
you receive source code or can get it if you want it, that you can
change the software or use pieces of it in new free programs; and that
you know you can do these things.

To protect your rights, we need to make restrictions that forbid anyone
to deny you these rights or to ask you to surrender the rights. These
restrictions translate to certain responsibilities for you if you
distribute copies of the software, or if you modify it.

For example, if you distribute copies of a such a program, whether
gratis or for a fee, you must give the recipients all the rights that
you have. You must make sure that they, too, receive or can get the
source code. And you must tell them their rights.

We protect your rights with two steps: (1) copyright the software, and
(2) offer you this license which gives you legal permission to copy,
distribute and/or modify the software.

Also, for each author's protection and ours, we want to make certain
that everyone understands that there is no warranty for this free
software. If the software is modified by someone else and passed on, we
want its recipients to know that what they have is not the original, so
that any problems introduced by others will not reflect on the original
authors' reputations.

The precise terms and conditions for copying, distribution and
modification follow.

\subsubsection{GNU GENERAL PUBLIC LICENSE}

\subsubsection{TERMS AND CONDITIONS FOR COPYING, DISTRIBUTION AND
MODIFICATION}

\textbf{0.} This License Agreement applies to any program or other work
which contains a notice placed by the copyright holder saying it may be
distributed under the terms of this General Public License. The
``Program'', below, refers to any such program or work, and a ``work
based on the Program'' means either the Program or any work containing
the Program or a portion of it, either verbatim or with modifications.
Each licensee is addressed as ``you''.

\textbf{1.} You may copy and distribute verbatim copies of the Program's
source code as you receive it, in any medium, provided that you
conspicuously and appropriately publish on each copy an appropriate
copyright notice and disclaimer of warranty; keep intact all the notices
that refer to this General Public License and to the absence of any
warranty; and give any other recipients of the Program a copy of this
General Public License along with the Program. You may charge a fee for
the physical act of transferring a copy.

\textbf{2.} You may modify your copy or copies of the Program or any
portion of it, and copy and distribute such modifications under the
terms of Paragraph 1 above, provided that you also do the following:

\textbf{a)} cause the modified files to carry prominent notices stating
that you changed the files and the date of any change; and

\textbf{b)} cause the whole of any work that you distribute or publish,
that in whole or in part contains the Program or any part thereof,
either with or without modifications, to be licensed at no charge to all
third parties under the terms of this General Public License (except
that you may choose to grant warranty protection to some or all third
parties, at your option).

\textbf{c)} If the modified program normally reads commands
interactively when run, you must cause it, when started running for such
interactive use in the simplest and most usual way, to print or display
an announcement including an appropriate copyright notice and a notice
that there is no warranty (or else, saying that you provide a warranty)
and that users may redistribute the program under these conditions, and
telling the user how to view a copy of this General Public License.

\textbf{d)} You may charge a fee for the physical act of transferring a
copy, and you may at your option offer warranty protection in exchange
for a fee.

Mere aggregation of another independent work with the Program (or its
derivative) on a volume of a storage or distribution medium does not
bring the other work under the scope of these terms.

\textbf{3.} You may copy and distribute the Program (or a portion or
derivative of it, under Paragraph 2) in object code or executable form
under the terms of Paragraphs 1 and 2 above provided that you also do
one of the following:

\textbf{a)} accompany it with the complete corresponding
machine-readable source code, which must be distributed under the terms
of Paragraphs 1 and 2 above; or,

\textbf{b)} accompany it with a written offer, valid for at least three
years, to give any third party free (except for a nominal charge for the
cost of distribution) a complete machine-readable copy of the
corresponding source code, to be distributed under the terms of
Paragraphs 1 and 2 above; or,

\textbf{c)} accompany it with the information you received as to where
the corresponding source code may be obtained. (This alternative is
allowed only for noncommercial distribution and only if you received the
program in object code or executable form alone.)

Source code for a work means the preferred form of the work for making
modifications to it. For an executable file, complete source code means
all the source code for all modules it contains; but, as a special
exception, it need not include source code for modules which are
standard libraries that accompany the operating system on which the
executable file runs, or for standard header files or definitions files
that accompany that operating system.

\textbf{4.} You may not copy, modify, sublicense, distribute or transfer
the Program except as expressly provided under this General Public
License. Any attempt otherwise to copy, modify, sublicense, distribute
or transfer the Program is void, and will automatically terminate your
rights to use the Program under this License. However, parties who have
received copies, or rights to use copies, from you under this General
Public License will not have their licenses terminated so long as such
parties remain in full compliance.

\textbf{5.} By copying, distributing or modifying the Program (or any
work based on the Program) you indicate your acceptance of this license
to do so, and all its terms and conditions.

\textbf{6.} Each time you redistribute the Program (or any work based on
the Program), the recipient automatically receives a license from the
original licensor to copy, distribute or modify the Program subject to
these terms and conditions. You may not impose any further restrictions
on the recipients' exercise of the rights granted herein.

\textbf{7.} The Free Software Foundation may publish revised and/or new
versions of the General Public License from time to time. Such new
versions will be similar in spirit to the present version, but may
differ in detail to address new problems or concerns.

Each version is given a distinguishing version number. If the Program
specifies a version number of the license which applies to it and ``any
later version'', you have the option of following the terms and
conditions either of that version or of any later version published by
the Free Software Foundation. If the Program does not specify a version
number of the license, you may choose any version ever published by the
Free Software Foundation.

\textbf{8.} If you wish to incorporate parts of the Program into other
free programs whose distribution conditions are different, write to the
author to ask for permission. For software which is copyrighted by the
Free Software Foundation, write to the Free Software Foundation; we
sometimes make exceptions for this. Our decision will be guided by the
two goals of preserving the free status of all derivatives of our free
software and of promoting the sharing and reuse of software generally.

\textbf{NO WARRANTY}

\textbf{9.} BECAUSE THE PROGRAM IS LICENSED FREE OF CHARGE, THERE IS NO
WARRANTY FOR THE PROGRAM, TO THE EXTENT PERMITTED BY APPLICABLE LAW.
EXCEPT WHEN OTHERWISE STATED IN WRITING THE COPYRIGHT HOLDERS AND/OR
OTHER PARTIES PROVIDE THE PROGRAM ``AS IS'' WITHOUT WARRANTY OF ANY
KIND, EITHER EXPRESSED OR IMPLIED, INCLUDING, BUT NOT LIMITED TO, THE
IMPLIED WARRANTIES OF MERCHANTABILITY AND FITNESS FOR A PARTICULAR
PURPOSE. THE ENTIRE RISK AS TO THE QUALITY AND PERFORMANCE OF THE
PROGRAM IS WITH YOU. SHOULD THE PROGRAM PROVE DEFECTIVE, YOU ASSUME THE
COST OF ALL NECESSARY SERVICING, REPAIR OR CORRECTION.

\textbf{10.} IN NO EVENT UNLESS REQUIRED BY APPLICABLE LAW OR AGREED TO
IN WRITING WILL ANY COPYRIGHT HOLDER, OR ANY OTHER PARTY WHO MAY MODIFY
AND/OR REDISTRIBUTE THE PROGRAM AS PERMITTED ABOVE, BE LIABLE TO YOU FOR
DAMAGES, INCLUDING ANY GENERAL, SPECIAL, INCIDENTAL OR CONSEQUENTIAL
DAMAGES ARISING OUT OF THE USE OR INABILITY TO USE THE PROGRAM
(INCLUDING BUT NOT LIMITED TO LOSS OF DATA OR DATA BEING RENDERED
INACCURATE OR LOSSES SUSTAINED BY YOU OR THIRD PARTIES OR A FAILURE OF
THE PROGRAM TO OPERATE WITH ANY OTHER PROGRAMS), EVEN IF SUCH HOLDER OR
OTHER PARTY HAS BEEN ADVISED OF THE POSSIBILITY OF SUCH DAMAGES.

\subsubsection{END OF TERMS AND CONDITIONS}

\subsubsection{Appendix: How to Apply These Terms to Your New Programs}

If you develop a new program, and you want it to be of the greatest
possible use to humanity, the best way to achieve this is to make it
free software which everyone can redistribute and change under these
terms.

To do so, attach the following notices to the program. It is safest to
attach them to the start of each source file to most effectively convey
the exclusion of warranty; and each file should have at least the
``copyright'' line and a pointer to where the full notice is found.

\begin{verbatim}
<one line to give the program's name and a brief idea of what it does.>
Copyright (C) 19yy <name of author>

    This program is free software; you can redistribute it and/or modify
    it under the terms of the GNU General Public License as published by
    the Free Software Foundation; either version 1, or (at your option)
    any later version.

    This program is distributed in the hope that it will be useful,
    but WITHOUT ANY WARRANTY; without even the implied warranty of
    MERCHANTABILITY or FITNESS FOR A PARTICULAR PURPOSE.  See the
    GNU General Public License for more details.

    You should have received a copy of the GNU General Public License
    along with this program; if not, write to the Free Software
    Foundation, Inc., 675 Mass Ave, Cambridge, MA 02139, USA.
\end{verbatim}

Also add information on how to contact you by electronic and paper mail.

If the program is interactive, make it output a short notice like this
when it starts in an interactive mode:

\begin{verbatim}
Gnomovision version 69, Copyright (C) 19xx name of author Gnomovision
comes with ABSOLUTELY NO WARRANTY; for details type `show w'. This is
free software, and you are welcome to redistribute it under certain
conditions; type `show c' for details.
\end{verbatim}

The hypothetical commands `show w' and `show c' should show the
appropriate parts of the General Public License. Of course, the commands
you use may be called something other than `show w' and `show c'; they
could even be mouse-clicks or menu items--whatever suits your program.

You should also get your employer (if you work as a programmer) or your
school, if any, to sign a ``copyright disclaimer'' for the program, if
necessary. Here a sample; alter the names:

\begin{verbatim}
Yoyodyne, Inc., hereby disclaims all copyright interest in the program
`Gnomovision' (a program to direct compilers to make passes at
assemblers) written by James Hacker.

<signature of Ty Coon>, 1 April 1989
Ty Coon, President of Vice
\end{verbatim}

That's all there is to it!

\end{document}
